\documentclass{rkim-resume}

\name{Russell Kim}
\email{rkim@xaipient.com}
\phone{(+1) 609-356-8385}
\github{russellkim98}
\linkedin{russellkim98}

\usepackage{CormorantGaramond}

\begin{document}

\maketitle%

\section{Professional \\ Experience}

\subsection{XaiPient}[New York, NY]
\begin{positions}
  \entry{ML Research Engineer}{Jun 2020~--~Present}
\end{positions}

\begin{itemize}
  \item Advisor: Prof.~Somesh Jha (\href{http://pages.cs.wisc.edu/~jha/}{http://pages.cs.wisc.edu/~jha/}).
  \item Researched and engineered attributions of sparse features for deep neural networks, as well as contributing to the core API for explainability AI products.
  \item Developed and integrated a gradient descent based Counterfactual explanation system which iterates using Tensorflow models as data controllers and returns class boundary probabilities based on Wacher et al.(2017)
\end{itemize}

\subsection{NLMatics}[New York, NY]
\begin{positions}
  \entry{NLP Research Intern}{Feb 2020~--~May 2020}
\end{positions}

\begin{itemize}
  \item Used principles from classic NLP, graph theory, bayesian probabilistic inference, bioinformatics, active learning, reinforcement learning and deep learning to create text sentiment analysis tools on large amounts of text data.
  \item Developed core API calls that implement an abstractive text summarizer based on a neural sequence to sequence model
  \item Worked with research training datasets such as GLUE and create self labeled datasets from publicly available corpuses based on in-house written algorithms such as siamese neural networks or deep Q-Learning.
\end{itemize}

\subsection{Princeton University}[Princeton, NJ]
\begin{positions}
  \entry{Undergraduate Research Fellow/ Keller Center}{Feb 2019~--~Sept 2019}
\end{positions}

\begin{itemize}
  \item Advisor: Prof.~Frederick Wherry (\href{https://sociology.princeton.edu/people/frederick-wherry}{https://sociology.princeton.edu/people/frederick-wherry})
  \item Time series analysis of credit card delinquencies based on financial data provided by the Brooklyn Financial Clinic and the Consumer Financial Protection Bureau.
  \item Implemented a web spider on Facebook\@ and created keyword network graph using a co-occurrence matrix to determine customer sentiment.
  \item Resulted in data contributed towards several research papers.
\end{itemize}

\subsection{Point72 Asset Management, LP}[New York, NY]
\begin{positions}
  \entry{Aperio Data Modeling Intern}{Jun 2018~--~Aug 2018}
\end{positions}

\begin{itemize}
  \item Worked under David Loaiza, PhD, on home construction company sales data and forecasting volatility trends
  \item Designed and implemented an automated script that would webscrape 12 different home construction websites periodically and store them in SQL databases.
  \item Developed an ARIMA time series forecasting the number of foreclosures and mortgage cancellations by exploring default trends and volatility in similar bunches of locations.
  \item Gave specific recomendations to change in project direction based on analysis over 8 years of home building data.
\end{itemize}

\section{Education}

\subsection{Princeton University}[Princeton, NJ]
\vspace{-\parskip}%
\begin{itemize}[label={}]
  \item B.S.E\ in ORFE with an emphasis on applied math and optimization\printdate{Sept 2016~--~Jun 2020}
  \item GPA: 3.7/4.0
  \item Thesis: {A Reinforcement Learning Based Approach to Pricing and Hedging Financial Derivatives}\href{https://github.com/russellkim98/Thesis/blob/master/Russell-Kim-Thesis.pdf}{\\(https://git.io/Jfrv7)}
  \item Advisor: Prof.~Mete Soner (\href{https://soner.princeton.edu/}{https://soner.princeton.edu/})
\end{itemize}
\section{Skills}


\begin{description}
  \item[Programming]: Python, Matlab, \LaTeX, Bash, R, Java
  \item[Frameworks]: Tensorflow, PyTorch, OpenAI Gym, NLTK, spaCy, Keras, Gensim
  \item[Languages]: English, Korean, French
\end{description}


\section{Research \\ Interests}

Topics: Nonparametric reinforcement learning, stochastic systems under closed MC simulation, derivatives pricing, convexity in explainability AI.

\begin{itemize}
  \item Nonparametric discrete definition of financial derivatives.
  \item Confidence bounding in boundary decision problems with sparse input features.
  \item One-shot learning of complex objects using siamese neural networks.
\end{itemize}


\end{document}
